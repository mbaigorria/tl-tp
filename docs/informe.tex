\documentclass{article}

\usepackage[utf8]{inputenc}
\usepackage[spanish]{babel}

\usepackage{caratula}

\usepackage{subcaption}
\usepackage{graphicx}
\usepackage{dirtytalk}
\usepackage{enumerate}

\usepackage{amssymb}
\usepackage{mathtools}
\usepackage{amsmath}
\usepackage{amsthm}

\usepackage{algorithm}
\usepackage{algpseudocode}
\usepackage{listingsutf8}

\usepackage{float}
\floatplacement{figure}{h!}

\usepackage{geometry}
\usepackage{fixltx2e}
\usepackage{wrapfig}
\usepackage{cite}
\usepackage{dsfont}
\usepackage{ulem}
\usepackage{xcolor}

\usepackage[space]{grffile}

\geometry{
 a4paper,
 total={210mm,297mm},
 left=30mm,
 right=30mm,
 top=30mm,
 bottom=30mm,
}
 
\usepackage{booktabs}

\newtheorem{theorem}{Teorema}[section]
\newtheorem{corollary}{Corolario}[theorem]
\newtheorem{lemma}{Lema}[theorem]
 
\theoremstyle{definition}
\newtheorem{definition}{Definición}[section]
 
\theoremstyle{remark}
\newtheorem*{remark}{Observación}
 
\begin{document}
% Estos comandos deben ir antes del \maketitle
\materia{Teoria de Lenguajes} % obligatorio

\titulo{TP1}
\subtitulo{NoSQL \\ \today}
\grupo{Grupo: Parseamela Gramatica}
 
\integrante{Mauro Cherubini}{835/13}{cheru.mf@gmail.com}
\integrante{Martin Baigorria}{575/14}{martinbaigorria@gmail.com}
 
\maketitle

\tableofcontents

\section{Gramatica}

La siguiente gramatica corresponde al lenguaje Dibu:

\begin{equation}
\begin{aligned}
\textcolor{blue}{INST} & \rightarrow & id \; \textcolor{red}{ARGL} \; \textcolor{blue}{INST}  \; | \; \lambda \\
\textcolor{red}{ARGL} & \rightarrow & ident = VAL \\
     & \quad | &  id = VAL, ARGL \\
VAL  & \rightarrow & num | string \\
     & \quad | & (VAL, VAL) | [LVAL] \\
LVAL & \rightarrow & LVAL, VAL | VAL
\end{aligned}
\end{equation}

\end{document}